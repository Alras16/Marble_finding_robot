\documentclass[../Head/Main.tex]{subfiles}
\begin{document}
\subsubsection{Line detection}
\paragraph{Least Square Method}
The Total Least Square method uses the normal parametrization of a line in polar coordinates, which is given by the following formula:
\begin{align}
    l:~~~~r = x\cdot\cos(\alpha)+y\cdot\sin(\alpha)
\end{align}
where r represents the distance from the origin to the closest point on the line and α is the angle between the x-axis and the plane normal. This method involves a determination of the orthogonal distance (the shortest distance) from a point $(x,y)$ to a line $l$ (see figure xx). The normal parametrization of the line $l_i$ is given by the following formula:
\begin{align}
    l_i:~~~~r_i = x_i\cdot\cos(\alpha)+y_i\cdot\sin(\alpha)
\end{align}
{\color{red} Missing figure} \par
The separation between those two lines ($l$ and $l_i$) is given by the difference $d_i=r_i-r$, since both lines have the same α. This means that the orthogonal distance can be described using the following formula:
\begin{align}
    d_i = x_i\cdot\cos(\alpha)+y_i\cdot\sin(\alpha) - r
\end{align}
This only applies if we assume that there is no noise on the measurements. \par
The Total Least Square Method solves the problem of fitting a straight line to a dataset of points p with n measurements having errors. The problem of fitting a line can be determined using the following sum:
\begin{align}
    \chi^2\left(l, z_1, ..., z_n\right) = \sum_{i = 1}^{n} \left[\frac{\left(x_k - X_k\right)^2}{u_{x, k}^{2}} + \frac{\left(y_k - Y_k\right)^2}{u_{y, k}^{2}}\right]
\end{align}
where $(x_k,y_k)$ are the points coordinates with corresponding uncertainties $(u_{x, k},u_{y, k})$ and $(X_k,Y_k)$ denote its corresponding point of the straight line l. In the case of fitting the best line to the dataset, minimizes the expression for $\chi^2$ by setting $u_{x, k}=u_{y, k}=\sigma$ and $k=1,…,n$. This reduces the problem to the Total Least Square method and minimizing is equal to minimizing the orthogonal distance of the measurements to the fitting line. Therefore, in the case  of fitting the best line minimizes the expression above to the following:
\begin{align}
    \chi^2\left(l; Z\right) &= \sum_{i = 1}^{n} \frac{d_i^2}{\sigma^2} \\
    &= \sum_{i = 1}^{n}\frac{\left(x_i\cos(\alpha) + y_i\sin⁡(\alpha)-r\right)^2}{\sigma^2}\\
    &= \frac{1}{\sigma^2}\cdot\sum_{i = 1}^{n}\left(x_i\cos(\alpha) + y_i\sin⁡(\alpha)-r\right)^2
\end{align}
A condition for minimizing $\chi^2$ is done by solving the nonlinear equation system with respect to each of the two line parameters ($r$ and $\alpha$)
\begin{align}
    \frac{\partial\chi^2}{\partial r} = 0 \hspace{1.5cm} \frac{\partial\chi^2}{\partial\alpha} = 0
\end{align}
The solution of this nonlinear equation system is determined to the following:
\begin{align}
	r &= \overline{X}\cos(\alpha)+\overline{Y}\sin(\alpha) \\
    \alpha &= \frac{1}{2}\arctan\left(\frac{-2\sum_{i=1}^{n}\left[\left(x_i-\overline{X}\right) - \left(y_i-\overline{Y}\right)\right]}{\sum_{i = 1}^{n}\left[\left(x_i-\overline{X}\right)^2 - \left(y_i-\overline{Y}\right)^2\right]}\right)
\end{align}
where $\overline{x}$ and $\overline{y}$ are the means of $x$ and $y$.
\paragraph{Line Extraction Algorithms}
It is usually important for a mobile robot to know its environment. There are several reasons for that, one is that robot must know the location of the obstacles relative to it to avoid driving into them. The environment is predefined in this project, and only has walls as obstacles. These walls can be detected using the robots Lidar sensor and a line extraction algorithm, since all obstacles in the Lidar data can be presented as a straight line. Here, there are several line extraction techniques to choose from.
We have chosen to implement the incremental line extraction algorithm, because of its simplicity. The incremental algorithm is shown in table xx.\par
\begin{figure}[H]
	\centering
	\begin{tabular}{c l}
		\hline
		\multicolumn{2}{l}{\textbf{Table							\ref{tab:Incremental}: Incremental Algorithm}}  			\\ \hline
		1 & Start by the first 2 points, construct a 				line \\
		2 & Add the next point to the current line model 		\\
		3 & Recompute the line parameters \\
		4 & If it satisfies line condition (go to 2) \\
		5 & Otherwise, put back the last point, 					recompute the line parameters, return the line  			\\
		6 & Continue to the next 2 points, go to 2 \\ 				\hline
	\end{tabular}
	\caption{Description of the Incremental algorithm}
	\label{tab:Incremental}
\end{figure}
This algorithm implements the Total Least Square method then computing the line parameters. Furthermore, the line model consists of points, which all must comply some line conditions. In that case all points must comply these line conditions. The first condition is a threshold for the angle between the previous and current line model (see figure xx). The threshold is defined to be the following:
\begin{align*}
	\theta_{max} = 0,0025
\end{align*}
The second condition is the angle between two points relative to the robot location (see figure yx). This angle should be greater than this difference, but not twice as great, since this condition should separate the points into two lines, if one point is missing on the list (see figure yy). This angle is therefore defined to be the following:
\begin{align*}
	\Delta\theta = \left(\theta_0 - \theta_1\right)\cdot 1,25
\end{align*}
{\color{red} Missing 3 figures} \par
\end{document}