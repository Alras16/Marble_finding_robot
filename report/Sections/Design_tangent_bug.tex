\documentclass[../Head/Main.tex]{subfiles}
\begin{document}
\subsection{Tangent bug}

The tangent bug algorithm is a sensor based planer which relies on inputs from a sensor to determine whether it should continue towards a  specified location $q_{goal}$ or to follow an obstacle $O_i$ until a free path is available. The advantage of using this algorithm is the computational minimization and the fact that it only needs a start and end point. The reason for using this algorithm is the benefits just explained and that it will always try to achieve a shortest path solution to the goal location. Pseudo code for the implementation for the tangent bug algorithm is shown below.        

\begin{Pseudo}{Tangent Bug Algorithm}{}
	\begin{Indentation}
		\item \textbf{Input:} A robot with a range sensor
		\item \textbf{Output:} A path to the $q_{goal}$	or a conclusion no such path exist 
		\begin{Indentation}
		
		\item \textbf{while} True \textbf{do}
			
			\begin{Indentation}
			
			\item \textbf{repeat}
			
				\begin{Indentation}

				\item Continuously move toward the point \textit{n} $\epsilon$ $\left(T,~O_i\right)$ which minimizes $d(x,~n) + d \left(n,~q_{goal}\right)$
				
				\item \textbf{until}
				\begin{itemize}
					\item The goal is encountered \textbf{or}
					\item The direction that minimizes ${d(x,n)}$ +d$\left(n,~q_{goal}\right)$ begins to increase d$\left(x,~q_{goal}\right)$ so the robot detects a local minimum d$\left(x,~O_i\right)$ + 
					d$\left(O_i,~q_{goal}\right)$ on the boundary \textit{i}.
					\item[ ]		
				\end{itemize}
				Now choose the boundary following direction which continuous in the same direction as the most recent motion-to-goal direction. 
				
				
				\item \textbf{repeat}
				\item Continuously update $d_{reach}$, $d_{followed}$ and $O_i$. 
				\item Continuously moves toward n $\epsilon$ \{$O_i$\} that is the chosen boundary direction.  
				
				\item \textbf{until}
				
				\begin{itemize}
				\item The goal is reached 
				\item The robot completes a cycle around the obstacle in which case the goal cannot be reached. 
				\item $d_{reach} < d_{followed}$
				
				\end{itemize}	
														
				\end{Indentation}			
			
			\end{Indentation}
		\item \textbf{end while}
		\end{Indentation}

	\end{Indentation}
	
\end{Pseudo}

It has been decided to make small changes to the algorithm and make it greedy. Instead of following the boundary $d_{reach} < d_{followed}$, the robot will follow that obstacle that minimizes ${d(x,n)}$ + d$\left(n,~q_{goal}\right)$. The reason for doing this is to reduce the path from a giving point to a target location.  

Because the point giving to the tangent bug is relative to the world frame in gazebo a mapping between to configurations is needed. The robot is shifted $\pi$ radians relative to the world frame which give the following transformation matrix 

\begin{align}
	T = \begin{bmatrix} 
	   cos(\theta-\frac{\pi}{2}) &  -sin(\theta-\frac{\pi}{2}) & robotPos.x \\ 
	    sin(\theta-\frac{\pi}{2}) &  cos(\theta-\frac{\pi}{2}) & robotPos.y \\
	    0 & 0 & 1					
	\end{bmatrix} 
\end{align}

\begin{align}
	p^{robot} = T^{-1} \cdot {p^{world}}
\end{align}
\label{Transformations matrix}

Now by calculating equation \ref{Transformations matrix} one can get the desired orientation to the target location from $p_{robot}$ by using atan2.


\end{document}