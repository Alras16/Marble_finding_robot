\documentclass[../Head/Main.tex]{subfiles}
\begin{document}
\subsection{Q-learning}
Q-learning was implemented by designing a class containing the functionality of holding the state space and update the Q-matrix. This class was named \texttt{q\_learning}.\par 
Due to the definition of the state, each room will have 2\textsuperscript{number of rooms} number of states, corresponding to the number of possible combinations of how the rooms could be visited. This will result in a very large Q-matrix. 
This could be overcome by splitting the Q-matrix into several smaller Q-matrices and then keep track of the mapping between them.\\
The Q-matrix would consist of number of base states by the number of actions which would be a 6 x 6 matrix with 5 rooms (including an initial state)


\begin{table}[H]
	\centering
	\begin{tabular}{|l|r|r|r|r|r|r|}
	\hline
     & \multicolumn{1}{l|}{\textbf{Start}} & \multicolumn{1}{l|}{\textbf{Room 1}} & \multicolumn{1}{l|}{\textbf{Room 2}} & \multicolumn{1}{l|}{\textbf{Room 3}} & \multicolumn{1}{l|}{\textbf{Room 4}} & \multicolumn{1}{l|}{\textbf{Room 5}} \\ \hline
\textbf{Start}  & 0                                   & {\color{red} -100}          & {\color{red} -100}          & 13.33971                             & {\color{red} -100}          & {\color{red} -100}          \\ \hline
\textbf{Room 1} & {\color{red} -100}         & 0                                    & 8.681309                             & {\color{red} -100}          & {\color{red} -100}          & {\color{red} -100}          \\ \hline
\textbf{Room 2} & {\color{red} -100}         & 0.405394                             & 0                                    & 10.74771                             & {\color{red} -100}          & {\color{red} -100}          \\ \hline
\textbf{Room 3} & {\color{red} -100}         & {\color{red} -100}          & 7.757309                             & 0                                    & 7.837891                             & 17.456                               \\ \hline
\textbf{Room 4} & {\color{red} -100}         & {\color{red} -100}          & {\color{red} -100}          & 11.01171                             & 0                                    & {\color{red} -100}          \\ \hline
\textbf{Room 5} & {\color{red} -100}         & {\color{red} -100}          & {\color{red} -100}          & 10.79571                             & {\color{red} -100}          & 0                                    \\ \hline
	\end{tabular}
	\caption{Rewards for all state-action combinations given that no rooms have been visited and initial state is room 3}
	\label{tab:reward_matrix_5_room}
\end{table}

\end{document}