\documentclass[../Head/Main.tex]{subfiles}
\begin{document}
\subsubsection{Sensor based planner}
The tangent bug algorithm is a sensor based planer which relies on inputs from a sensor to determine whether it should continue towards a  specified location $q_{goal}$ or to follow an obstacle $O_i$ until a free path is available. The advantage of using this algorithm is the computational minimisation and the fact that it only needs a start and end point. The reason for using this algorithm is the benefits just explained and that it will always try to achieve a shortest path solution to the goal location. Pseudo code for the implementation for the tangent bug algorithm is shown below. (\cite[23-37]{ROB})    
\subfile{../Pseudo/Tangent_bug_algorithm}
It has been decided to make small changes to the algorithm and make it greedy. Instead of following the boundary $d_{reach} < d_{followed}$, the robot will follow that obstacle that minimises ${d(x,n)}$ + d$\left(n,~q_{goal}\right)$. The reason for doing this is to reduce the path from a giving point to a target location. \par
Because the point giving to the tangent bug is relative to the world frame in Gazebo a mapping between to configurations is needed. The robot is shifted $\pi$ radians relative to the world frame which give the following transformation matrix. (\cite[60-66]{ROB})
\begin{equation}
	T = \begin{bmatrix} 
	   cos(\theta-\frac{\pi}{2}) &  -sin(\theta-\frac{\pi}{2}) & robotPos.x \\ 
	    sin(\theta-\frac{\pi}{2}) &  cos(\theta-\frac{\pi}{2}) & robotPos.y \\
	    0 & 0 & 1					
	\end{bmatrix}
\end{equation}
\begin{equation}\label{eq:world_transformation}
	p^{robot} = T^{-1} \cdot {p^{world}}
\end{equation}	
Now by calculating equation (\ref{eq:world_transformation}) one can get the desired orientation to the target location from $p_{robot}$ by using atan2.
\end{document}