\documentclass[../Head/Main.tex]{subfiles}
\begin{document}
\subsection{Q-learning}
In order to effectively search the environment and collect marbles, a good search strategy must be found. This can be done by utilising reinforcement learning. By using reinforcement learning, the robot can learn from its experience and obtain a good strategy for navigating the environment.\par
By using a Temporal-Difference learning strategy the optimal action-value function can be estimated by every move taken unlike a Monte Carlo strategy where an episode terminates before any learning is obtained. In some cases with long episodes the Monte Carlo strategy is considered too slow.\\
Generally there are two categories of Temporal-Difference learning; on-policy and off-policy methods. One of the advantages of an off-policy over an on-policy method are that the action-value function can be estimated independent from the policy being used. The policy only influences which state-action pairs that are visited and updated.\par 
Based on this Q-learning are chosen, the Q-learning method builds on the following update function for updating the action-value function (Q-values).
\begin{equation}\label{eq:q_update_func}
Q\left(S_t,A_t\right) \leftarrow Q\left(S_t,A_t\right) + \alpha\left[R_{t+1}+\gamma\max_a Q\left(S_{t+1},a\right)-Q\left(S_t,A_t\right)\right]
\end{equation}
The update function for Q-learning consists of the old value for a given state-action combination plus a scaled difference between the old value, the immediate reward and the maximal value for the next state. The learning rate are denoted $\alpha$ and ranging from 0-1 preferable closer to 0, in order to not to base the policy on this action only. $\gamma$ denotes the discount factor and are also ranging from 0-1, preferable closer to 1, to ensure that future actions matter.\par
In the box below, the algorithm for Q-learning can be seen.
\begin{Pseudo}{Q-learning}{}
\begin{Indentation}
    \item Algorithm parameters: step size: $\alpha\in$ [0,1], small  $\epsilon$ > 0 \vspace{-2pt}
    \item Initialise $Q(s,a)$, for all $s\in S^{+}$, $a\in A(s)$, arbitrarily except that $Q(terminal,\dot)=0$
    \item Loop for each episode: \vspace{-2pt}
    \begin{Indentation}
        \item Initialise $S$ \vspace{-2pt}
        \item Loop for each step of episode: \vspace{-2pt}
        \begin{Indentation}
            \item Choose $A$ from $S$ using policy derived from $Q(e.g., \epsilon-greedy)$ \vspace{-2pt}
            \item Take action $A$, observe $R$, $S^{'}$ \vspace{-1pt}
            \item $Q(S,A) \leftarrow Q(S,A) + \alpha \left[R + \gamma \max\limits_{a} Q(S',a) - Q(S,A)\right]$
            \item $S\leftarrow S^{'}$ \vspace{-2pt}
        \end{Indentation}
        \item until $S$ is terminal 
    \end{Indentation}
\end{Indentation}
\end{Pseudo}
\subsubsection{Definition of states}

\begin{itemize}
	\item[-] Room number
	\item[-] Vector of boolean values one for each room to know which rooms have been visited
	\item[-] Boolean for knowing if the state is terminal or not 
\end{itemize}

\begin{Pseudo}{$\epsilon$-greedy policy}{}
	\begin{Indentation}
		\item if random number < $\epsilon$ \vspace{-2pt}
			\begin{Indentation}
				\item return random action \vspace{-2pt}
			\end{Indentation}
		\item else \vspace{-2pt}
		\begin{Indentation}
			\item return policy action
		\end{Indentation}
	\end{Indentation}
\end{Pseudo}


\begin{Pseudo}{getNextAction()}{}
	\begin{Indentation}
		\item loop all actions for a given state \vspace{-2pt}
		\begin{Indentation}
			\item if Q-value is higher than maxValue
		\end{Indentation}
	\end{Indentation}
\end{Pseudo}
\end{document}